\documentclass[11pt, openright, a4paper, brazil, english, french, spanish, bibjustif, xcolor=table,aspectratio=169]{beamer}

%inserindo templates

\usetheme{Feather}

\usepackage{comment}
\usepackage{cmap}				% Mapear caracteres especiais no PDF
\usepackage{ifthen}				% Mapear caracteres especiais no PDF
\usepackage{setspace}				% Mapear caracteres especiais no PDF
\usepackage{lmodern}		
\usepackage[T1]{fontenc}	
\usepackage[utf8]{inputenc}
\usepackage{lastpage}
\usepackage{enumitem}	
\usepackage{indentfirst}	
\usepackage{color}			
\usepackage{graphicx}
%\usepackage[table]{xcolor}
\usepackage{tabularx}
\usepackage{booktabs}
\usepackage{enumerate}		
\usepackage{microtype} 		
\usepackage{multicol}
\usepackage{multirow}
\usepackage{lipsum}
\usepackage{booktabs}				
\usepackage{bold-extra}				% Mapear caracteres especiais no PDF
\usepackage[final]{pdfpages}
\usepackage{float}

%pacote para símbolos especiais
\usepackage{amssymb}

%pacote para fazer diagramas de venn

\usepackage{venndiagram}

%aumentei para fazer texto em fundo colorido
\usepackage{xcolor}
%aumentei a próxima linha
\usepackage{verbatim}

% \usepackage{subfig}
%\usepackage{subcaption}


\usepackage[brazilian,hyperpageref]{backref}	 % Paginas com as citações na bibl
\usepackage[alf]{abntex2cite}	% Citações padrão ABNT

\usepackage{caption}


\newtheorem{teo}{Teorema}

%espaçamento do parágrafo
	 
\setlength{\parskip}{0.2cm}


\begin{document}

\begin{frame}[t]{Funções}

\medskip

\begin{minipage}{\columnwidth}

\pause

\centering{

\resizebox{!}{.6cm}{

\textbf{Em função de}:} 

}

\pause

\begin{itemize}[label=$\bullet$]

\item ``de acordo com'',

\pause

\item ``em conformidade com'',

\pause

\item ``na dependência de'',

\pause

\item ``em resultado de''.

\end{itemize}


\end{minipage}

\end{frame}

\begin{frame}[t]{Funções}

\medskip

\begin{minipage}{\columnwidth}

Sendo $\textbf{A}$ e $\textbf{B}$ dois conjuntos não vazios e uma relação $\textbf{f}$ de $\textbf{A}$ em $\textbf{B}$, essa relação $\textbf{f}$ é uma função de $\textbf{A}$ em $\textbf{B}$ quando a cada elemento $\textbf{x}$ do conjunto $\textbf{A}$ está associado um e um só elemento $\textbf{y}$ do conjunto $\textbf{B}$.


\end{minipage}

\end{frame}


\begin{frame}[t]{Funções}


\end{frame}






\end{document}